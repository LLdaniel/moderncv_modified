%%%%%%%%%%%%%%%%%%%%%%%%%%%%%%%%%%%%%%%%%
% "ModernCV" CV and Cover Letter
% LaTeX Template
% Version 1.3 (29/10/16)
%
% This template has been downloaded from:
% http://www.LaTeXTemplates.com
%
% Original author:
% Xavier Danaux (xdanaux@gmail.com) with modifications by:
% Vel (vel@latextemplates.com)
%
% License:
% CC BY-NC-SA 3.0 (http://creativecommons.org/licenses/by-nc-sa/3.0/)
%
% Important note:
% This template requires the moderncv.cls and .sty files to be in the same 
% directory as this .tex file. These files provide the resume style and themes 
% used for structuring the document.
%
%%%%%%%%%%%%%%%%%%%%%%%%%%%%%%%%%%%%%%%%%

%----------------------------------------------------------------------------------------
%	PACKAGES AND OTHER DOCUMENT CONFIGURATIONS
%----------------------------------------------------------------------------------------

\documentclass[11pt,a4paper,sans]{moderncv} % Font sizes: 10, 11, or 12; paper sizes: a4paper, letterpaper, a5paper, legalpaper, executivepaper or landscape; font families: sans or roman
\usepackage[T1]{fontenc}
\usepackage[latin1]{inputenc}
\moderncvstyle{casual} % CV theme - options include: 'casual' (default), 'classic', 'oldstyle' and 'banking'
\moderncvcolor{blue} % CV color - options include: 'blue' (default), 'orange', 'green', 'red', 'purple', 'grey' and 'black'

\usepackage{lipsum} % Used for inserting dummy 'Lorem ipsum' text into the template
\usepackage{url}
\usepackage{tikz}
\usepackage{graphicx}
\usepackage[scale=0.8]{geometry} % Reduce document margins
\setlength{\hintscolumnwidth}{3.5cm} % Uncomment to change the width of the dates column
%\setlength{\makecvtitlenamewidth}{10cm} % For the 'classic' style, uncomment to adjust the width of the space allocated to your name

%----------------------------------------------------------------------------------------
%	NAME AND CONTACT INFORMATION SECTION
%----------------------------------------------------------------------------------------

\firstname{First name} % Your first name
\familyname{Last name} % Your last name

% All information in this block is optional, comment out any lines you don't need
\title{Lebenslauf}
\address{Steet}{City}
\mobile{+123456}
%\phone{(000) 111 1112}
%\fax{(000) 111 1113}
\email{email@mail.com}
%\homepage{staff.org.edu/~jsmith}{staff.org.edu/$\sim$jsmith} % The first argument is the url for the clickable link, the second argument is the url displayed in the template - this allows special characters to be displayed such as the tilde in this example
%\extrainfo{additional information}
\photo[110pt][0.2pt]{Bild} % The first bracket is the picture height, the second is the thickness of the frame around the picture (0pt for no frame)
%\quote{"A witty and playful quotation" - John Smith}

%----------------------------------------------------------------------------------------

\begin{document}
%----------------------------------------------------------------------------------------
%	COVER LETTER
%----------------------------------------------------------------------------------------

% To remove the cover letter, comment out this entire block

\clearpage

\recipient{Company}{Street\\123 City} % Letter recipient
\date{\today} % Letter date
\appsubject{application subject}% subject
\opening{Dear Sir or Madam,} % Opening greeting
\signature{0.65}{Unterschrift.jpg}
\closing{Mit freundlichen Gr��en,} % Closing phrase
\enclosure[Anlagen]{Lebenslauf, Zeugnis Master Of Science, Transcript Of Records Master, Zeugnis Bachelor Of Science} % List of enclosed documents

\makelettertitle % Print letter title

\lipsum[1-2] % Dummy text
\lipsum[4] % Dummy text

\makeletterclosing % Print letter signature

\newpage

%----------------------------------------------------------------------------------------
%	CURRICULUM VITAE
%----------------------------------------------------------------------------------------

\makecvtitle % Print the CV title
%
% define the height of the ending bar for section entries --> moved to moderncvbodyi.sty
%\newlength{\barheight}
%\setlength{\barheight}{0.95ex}
%\newlength{\missinglength}
%\newcommand{\sectionendingbar}{\hspace{\separatorcolumnwidth}\rule[\baseletterheight]{\maincolumnwidth-\separatorcolumnwidth-\missinglength}{\barheight}}
%\newcommand{\measuresection}[1]{\settowidth{\missinglength}{\sectionstyle{#1}}} % measure the current section length
%
%----------------------------------------------------------------------------------------
%	PERSONAL DATA
%----------------------------------------------------------------------------------------
\measuresection{Pers�nliche Daten}
\section{Pers�nliche Daten}
\cvitem{Name}{Name} 
\cvitem{Vorname}{Vorname}
\cvitem{Geburtsdatum}{01.01.2019}
\cvitem{Geburtsort}{City}
\cvitem{Anschrift}{Street, City}
\cvitem{Familienstand}{}
\cvitem{E-mail}{email@mail.com}
\cvitem{Telefon}{+123456}
%----------------------------------------------------------------------------------------
%	EDUCATION
%----------------------------------------------------------------------------------------
\measuresection{Ausbildung}
\section{Ausbildung}
\cventry{12/2019--12/2019}{ABC}{Uni}{City}{}{Abschlussnote: 0,7} % Arguments not required can be left empty
\cventry{12/2019--12/2019}{ACC}{Uni}{City}{}{Abschlussnote: 0,8}  % Arguments not required can be left empty
\cventry{12/2019--12/2019}{AAB}{Uni}{City}{}{}
\cventry{12/2019--07/2019}{ABB}{School}{City}{}{Abschlussnote: 0,7}

\measuresection{Wissenschaftliche Arbeiten}
\section{Wissenschaftliche Arbeiten}
\cventry{Masterarbeit}{Title}{Lehrstuhl}{Prof}{}{}
\cventry{Bachelorarbeit}{}{Lehrstuhl}{Prof}{}{}
%\cvitem{Title}{\emph{Money Is The Root Of All Evil -- Or Is It?}}
%\cvitem{Supervisors}{Professor James Smith \& Associate Professor Jane Smith}
%\cvitem{Description}{This thesis explored the idea that money has been the cause of untold anguish and suffering in the world. I found that it has, in fact, not.}
\newpage
%----------------------------------------------------------------------------------------
%	NEBENT�TIGKEITEN
%----------------------------------------------------------------------------------------
\measuresection{Nebent�tigkeiten}
\section{Nebent�tigkeiten}

\subsection{ABC}
\cventry{12/2019--12/2019}{Title}{What institution}{Uni}{City}{}

\subsection{ABB}
\cventry{12/2019 und 12/2019}{Title}{What}{Uni}{City}{}

%----------------------------------------------------------------------------------------
%	COMPUTER SKILLS SECTION
%----------------------------------------------------------------------------------------
\measuresection{IT Kenntnisse}
\section{IT Kenntnisse}
\cvitem{Programmiersprachen}{% ------------------------------------------------------------------
        \begin{tikzpicture}
                          {     \draw[fill=color2,color2] (0,1) rectangle (3,1.2);
                                \draw[fill=color3,color3](0,1) rectangle (2.4,1.2);
                                %
                                \draw [fill=white,white] (0.599,1) rectangle (0.601,1.2);
                                \draw [fill=white,white] (1.199,1) rectangle (1.201,1.2);
                                \draw [fill=white,white] (1.799,1) rectangle (1.801,1.2);
                                \draw [fill=white,white] (2.399,1) rectangle (2.401,1.2);
                                %
                                \node [above right] at (0,1.2) {C++};
                                \hspace{4cm}
                                \draw[fill=color2,color2] (0,1) rectangle (3,1.2);
                                \draw[fill=white,color3](0,1) rectangle (2.4,1.2);
                                %
                                \draw [fill=white,white] (0.599,1) rectangle (0.601,1.2);
                                \draw [fill=white,white] (1.199,1) rectangle (1.201,1.2);
                                \draw [fill=white,white] (1.799,1) rectangle (1.801,1.2);
                                \draw [fill=white,white] (2.399,1) rectangle (2.401,1.2);
                                %
                                \node [above right] at (0,1.2) {R};
                                \hspace{4cm}
                                \draw[fill=color2,color2] (0,1) rectangle (3,1.2);
                                \draw[fill=white,color3](0,1) rectangle (0.6,1.2);
                                %
                                \draw [fill=white,white] (0.599,1) rectangle (0.601,1.2);
                                \draw [fill=white,white] (1.199,1) rectangle (1.201,1.2);
                                \draw [fill=white,white] (1.799,1) rectangle (1.801,1.2);
                                \draw [fill=white,white] (2.399,1) rectangle (2.401,1.2);
                                %
                                \node [above right] at (0,1.2) {JavaScript};
                         }
         \end{tikzpicture}
}
\cvitem{GUI}{% ---------------------------------------------------------------------------------
         \begin{tikzpicture}
                          {     \draw[fill=color2,color2] (0,1) rectangle (3,1.2);
                                \draw[fill=color3,color3](0,1) rectangle (2.4,1.2);
                                %
                                \draw [fill=white,white] (0.599,1) rectangle (0.601,1.2);
                                \draw [fill=white,white] (1.199,1) rectangle (1.201,1.2);
                                \draw [fill=white,white] (1.799,1) rectangle (1.801,1.2);
                                \draw [fill=white,white] (2.399,1) rectangle (2.401,1.2);
                                %
                                \node [above right] at (0,1.2) {Assembler};
                         }
         \end{tikzpicture}
}
\cvitem{Datenanalyse}{
         \begin{tikzpicture}
                          {     \draw[fill=color2,color2] (0,1) rectangle (3,1.2);
                                \draw[fill=color3,color3](0,1) rectangle (2.4,1.2);
                                %
                                \draw [fill=white,white] (0.599,1) rectangle (0.601,1.2);
                                \draw [fill=white,white] (1.199,1) rectangle (1.201,1.2);
                                \draw [fill=white,white] (1.799,1) rectangle (1.801,1.2);
                                \draw [fill=white,white] (2.399,1) rectangle (2.401,1.2);
                                %
                                \node [above right] at (0,1.2) {LabView};
                                \hspace{4cm}
                                \draw[fill=color2,color2] (0,1) rectangle (3,1.2);
                                \draw[fill=white,color3](0,1) rectangle (2.4,1.2);
                                %
                                \draw [fill=white,white] (0.599,1) rectangle (0.601,1.2);
                                \draw [fill=white,white] (1.199,1) rectangle (1.201,1.2);
                                \draw [fill=white,white] (1.799,1) rectangle (1.801,1.2);
                                \draw [fill=white,white] (2.399,1) rectangle (2.401,1.2);
                                %
                                \node [above right] at (0,1.2) {OriginPro};
                                \hspace{4cm}
                                \draw[fill=color2,color2] (0,1) rectangle (3,1.2);
                                \draw[fill=white,color3](0,1) rectangle (1.8,1.2);
                                %
                                \draw [fill=white,white] (0.599,1) rectangle (0.601,1.2);
                                \draw [fill=white,white] (1.199,1) rectangle (1.201,1.2);
                                \draw [fill=white,white] (1.799,1) rectangle (1.801,1.2);
                                \draw [fill=white,white] (2.399,1) rectangle (2.401,1.2);
                                %
                                \node [above right] at (0,1.2) {Mathematica};
                         }
         \end{tikzpicture}
}
\cvitem{Betriebssysteme}{% -------------------------------------------------------------
         \begin{tikzpicture}
                          {     \draw[fill=color2,color2] (0,1) rectangle (3,1.2);
                                \draw[fill=color3,color3](0,1) rectangle (2.4,1.2);
                                %
                                \draw [fill=white,white] (0.599,1) rectangle (0.601,1.2);
                                \draw [fill=white,white] (1.199,1) rectangle (1.201,1.2);
                                \draw [fill=white,white] (1.799,1) rectangle (1.801,1.2);
                                \draw [fill=white,white] (2.399,1) rectangle (2.401,1.2);
                                %
                                \node [above right] at (0,1.2) {Debian, Ubuntu};
                                \hspace{4cm}
                                \draw[fill=color2,color2] (0,1) rectangle (3,1.2);
                                \draw[fill=white,color3](0,1) rectangle (2.4,1.2);
                                %
                                \draw [fill=white,white] (0.599,1) rectangle (0.601,1.2);
                                \draw [fill=white,white] (1.199,1) rectangle (1.201,1.2);
                                \draw [fill=white,white] (1.799,1) rectangle (1.801,1.2);
                                \draw [fill=white,white] (2.399,1) rectangle (2.401,1.2);
                                %
                                \node [above right] at (0,1.2) {Microsoft Windows};
                         }
         \end{tikzpicture}
}
\cvitem{Design/Bildbearbeitung}{
         \begin{tikzpicture}
                          {     \draw[fill=color2,color2] (0,1) rectangle (3,1.2);
                                \draw[fill=color3,color3](0,1) rectangle (2.4,1.2);
                                %
                                \draw [fill=white,white] (0.599,1) rectangle (0.601,1.2);
                                \draw [fill=white,white] (1.199,1) rectangle (1.201,1.2);
                                \draw [fill=white,white] (1.799,1) rectangle (1.801,1.2);
                                \draw [fill=white,white] (2.399,1) rectangle (2.401,1.2);
                                %
                                \node [above right] at (0,1.2) {Adobe Photoshop CC};
                                \hspace{4cm}
                                \draw[fill=color2,color2] (0,1) rectangle (3,1.2);
                                \draw[fill=white,color3](0,1) rectangle (2.4,1.2);
                                %
                                \draw [fill=white,white] (0.599,1) rectangle (0.601,1.2);
                                \draw [fill=white,white] (1.199,1) rectangle (1.201,1.2);
                                \draw [fill=white,white] (1.799,1) rectangle (1.801,1.2);
                                \draw [fill=white,white] (2.399,1) rectangle (2.401,1.2);
                                %
                                \node [above right] at (0,1.2) {Adobe InDesign CC};
                                \hspace{4cm}
                                \draw[fill=color2,color2] (0,1) rectangle (3,1.2);
                                \draw[fill=white,color3](0,1) rectangle (1.8,1.2);
                                %
                                \draw [fill=white,white] (0.599,1) rectangle (0.601,1.2);
                                \draw [fill=white,white] (1.199,1) rectangle (1.201,1.2);
                                \draw [fill=white,white] (1.799,1) rectangle (1.801,1.2);
                                \draw [fill=white,white] (2.399,1) rectangle (2.401,1.2);
                                %
                                \node [above right] at (0,1.2) {Adobe Illustrator CC};
                         }
         \end{tikzpicture}
}
\cvitem{Sonstige}{
         \begin{tikzpicture}
                          {     \draw[fill=color2,color2] (0,1) rectangle (3,1.2);
                                \draw[fill=color3,color3](0,1) rectangle (2.4,1.2);
                                %
                                \draw [fill=white,white] (0.599,1) rectangle (0.601,1.2);
                                \draw [fill=white,white] (1.199,1) rectangle (1.201,1.2);
                                \draw [fill=white,white] (1.799,1) rectangle (1.801,1.2);
                                \draw [fill=white,white] (2.399,1) rectangle (2.401,1.2);
                                %
                                \node [above right] at (0,1.2) {\LaTeX};
                                \hspace{4cm}
                                \draw[fill=color2,color2] (0,1) rectangle (3,1.2);
                                \draw[fill=white,color3](0,1) rectangle (3,1.2);
                                %
                                \draw [fill=white,white] (0.599,1) rectangle (0.601,1.2);
                                \draw [fill=white,white] (1.199,1) rectangle (1.201,1.2);
                                \draw [fill=white,white] (1.799,1) rectangle (1.801,1.2);
                                \draw [fill=white,white] (2.399,1) rectangle (2.401,1.2);
                                %
                                \node [above right] at (0,1.2) {LibreOffice};
                         }
         \end{tikzpicture}
}
\scriptsize
  \cvitem{}{($^\star$Kenntnis-Skala reicht von \textit{grundlegender Kenntnis} \begin{tikzpicture}\draw[fill=color2,color2] (0,1) rectangle (1.25,1.1);\draw[fill=white,color3](0,1) rectangle (0.25,1.1);\draw [fill=white,white] (0.2499,1) rectangle (0.2501,1.1);\draw [fill=white,white] (0.4999,1) rectangle (0.5010,1.1);\draw [fill=white,white] (0.7499,1) rectangle (0.7501,1.1);\draw [fill=white,white] (0.9999,1) rectangle (1.0010,1.1);\end{tikzpicture} bis \textit{Expertenkenntnis} \begin{tikzpicture}\draw[fill=color2,color2] (0,1) rectangle (1.25,1.1);\draw[fill=white,color3](0,1) rectangle (1.25,1.1);\draw [fill=white,white] (0.2499,1) rectangle (0.2501,1.1);\draw [fill=white,white] (0.4999,1) rectangle (0.5010,1.1);\draw [fill=white,white] (0.7499,1) rectangle (0.7501,1.1);\draw [fill=white,white] (0.9999,1) rectangle (1.0010,1.1);\end{tikzpicture})}
  \normalsize
%----------------------------------------------------------------------------------------
%	LANGUAGES SECTION
%----------------------------------------------------------------------------------------
\measuresection{Sprachkenntnisse}
\section{Sprachkenntnisse}

\cvitemwithcomment{lang1}{}{}
\cvitemwithcomment{lang2}{}{}

%----------------------------------------------------------------------------------------
%	INTERESTS SECTION
%----------------------------------------------------------------------------------------
\measuresection{Interessen}
\section{Interessen}
%\renewcommand{\listitemsymbol}{-~} % Changes the symbol used for lists
\cvitemwithcomment{kreativ}{ABC}{}
\cvitemwithcomment{sportlich}{ABB}{}
\cvitemwithcomment{systematisch}{ACC}{}
\cvitemwithcomment{sonstiges}{AAC}{}
%\cvlistdoubleitem{Piano}{Chess}
%\cvlistdoubleitem{Cooking}{Dancing}
%\cvlistitem{Running}

%----------------------------------------------------------------------------------------

\end{document}
